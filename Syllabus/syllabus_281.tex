\documentclass [12pt]{article}
\usepackage{palatino}
\usepackage{geometry}
\usepackage{amsmath, amsthm, amssymb}
\usepackage{natbib}
\usepackage{booktabs}
\usepackage{url}
\usepackage{booktabs}
\usepackage{threeparttable}
\usepackage{graphicx}
\usepackage{setspace}
\usepackage{subcaption}
\usepackage{comment}
%\usepackage{hyperref}
\usepackage[skip=0pt]{caption}

\usepackage{multirow}
\usepackage{rotating}
\usepackage{array}
\usepackage{float}

\newcommand{\mar}{1.0in}
\geometry{verbose,letterpaper,tmargin=\mar,bmargin=\mar,lmargin=\mar, rmargin=\mar} 

\renewcommand{\baselinestretch}{1.50}
\setlength{\bibsep}{1.2ex}
%\setlength{\parindent}{0pt}
\begin{document}
\begin{center}
\textbf{Econ 281\\
Special Topics in Economics\\
Spring 2022}
\end{center}


\noindent \textbf{Instructors} \\
Juan Herre\~no\\ 
jherrenolopera@ucsd.edu\\
Office Hours: Friday 10:30am - 12:00pm (Econ 212)\\

noindent Johannes Wieland\\ 
jfwieland@ucsd.edu\\
Office Hours: XX XX - XX  (Econ 309)\\


\noindent \textbf{Lectures} \\
Time: Tuesdays from 2:00pm to 4:50pm
Location: Econ 300.\\


\noindent \textbf{Course Description:} 



\noindent  \textbf{Readings}

\noindent The syllabus includes more papers than we expect the typical student to read. Readings marked with * are required.


\noindent  \textbf{Class Project}


\noindent  \textbf{Grading}

\noindent The first iteration for the class project will count for XX\% of the grade. The second iteration will count for XX\% of the grade. Class participation will count for XX\%.

\noindent  \textbf{Auditing}

\noindent We expect students who audit the class to participate and do the required readings. If you want to submit your class project and get feedback on it, you should take the class for grade.

\newpage

\section{Introduction to Empirics in Macroeconomics}

Inference with macroeconomic data is notoriously difficult due to reverse causality, small samples, and endogenous responses of economic policy attempting to stabilize the economy. In this lecture we will give an introduction to how to approach this problem using more disaggregated data, and measures of  \textit{shocks}, and heterogeneous \textit{exposure} to those shocks, or shift-share designs in short. We will cover the identifying assumptions of shift-share designs, and new methods developed to solve them. 

\section{Fiscal Multipliers}

What are the empirical challenges of measuring the size of the regional fiscal multiplier? We will apply our knowledge of shift-share designs in order to study the empirical challenges of estimating the multiplier. 

\section{Regional Aggregation I: The Case of Fiscal Multipliers}

Once we have recovered regional multipliers after increases in government expenditures or transfers, how can one recover the aggregate multipliers? In this lecture we will cover some of the papers that move from ``open-economy'' multipliers, into the aggregate multiplier.


\section{Other Regional Aggregation}

Moving from regional estimates to aggregate estimates goes beyond the case of fiscal multipliers, and extends to the determinants of inflation, the transmission of international shocks, and financial shocks. In this lecture we will cover some of the leading applications.


\section{Household and Firm Aggregation}


We have studied so far cases of aggregation of regional elasticities into national elasticities, the case of local GE effects. In this lecture we will study leading papers computing household-level or firm-level elasticities to economic shocks, and study how can one use these elasticities to learn about the national economy.


\section{HANK and TANK}

In this lecture we will study the leading models of heterogeneity of the New Keynesian tradition. The Two-Agent New Keynesian Model (TANK) and the Heterogeneous Agent New Keynesian Model (HANK). We will compare the transmission mechanism in these models compared to the Representative Agent New Keynesian Model (RANK).


\section{The Sequence Space}

We will learn how to write models in the Sequence Space, and learn a powerful tool to solve models with heterogeneity in discrete time with aggregate shocks.


\section{Macroeconomics of Consumption with Heterogeneity}

In this lecture we will study how consumption responses that are in line with the micro data impact the transmission of economic policy. We will discuss the challenges of incorporating heterogeneity in MPCs in macro models, the challenges of solving the models, and the implications this models have about the conduct of macro stabilization policy.

\section{Macroeconomics of Investment with Heterogeneity}

In this lecture we will study how investment responses that are in line with the micro data impact the transmission of economic policy. We will discuss the challenges of incorporating heterogeneity in propensities to invest in macro models, the challenges of solving the models, and the implications this models have about the conduct of macro stabilization policy.



\end{document}