\documentclass[english,xcolor=svgnames]{beamer}


\usepackage{mathptmx}
\usepackage[OT1]{fontenc}
% \usepackage[latin9]{inputenc}
\usepackage{amsmath}
\usepackage{amssymb}
\usepackage{amsthm}
\usepackage{mathrsfs}
\usepackage{amsfonts}
\usepackage{eurosym}
\usepackage{bm}

\usepackage{booktabs}
\usepackage{tabularx}
\usepackage{subcaption}
\usepackage[makeroom,thicklines]{cancel}

\usepackage{multirow}
\usepackage{rotating}
\usepackage{array}
\usepackage{float}



\makeatletter

 \newcommand\makebeamertitle{\frame{\maketitle}}%
 \AtBeginDocument{
   \let\origtableofcontents=\tableofcontents
 \def\tableofcontents{\@ifnextchar[{\origtableofcontents}{\gobbletableofcontents}}
   \def\gobbletableofcontents#1{\origtableofcontents}
 }
 
 \usetheme{Boadilla}
\setbeamertemplate{footline}[frame number]{}
\usefonttheme{structuresmallcapsserif}
\setbeamercolor{title}{fg=blue}
\setbeamercolor{frametitle}{fg=blue}
\setbeamercolor{caption name}{fg=blue}
\setbeamercovered{transparent}


\beamertemplatenavigationsymbolsempty

\usepackage{booktabs}
\usepackage{tabularx}
\renewcommand{\tabularxcolumn}[1]{>{\centering\arraybackslash}m{#1}}
%\newcolumntype{L}{>{\centering}X}
%\newcolumntype{H}{>{\lrbox0}c<{\endlrbox}@{}}

%\let\estinput=\input
%\newcommand{\estwide}[3]{
%          \vspace{.75ex}{
%               \begin{tabularx}
%               {\textwidth}{@{\hskip\tabcolsep\extracolsep\fill}l*{#2}{#3}}
%               \toprule
%               \estinput{#1}
%               \bottomrule
%               \addlinespace[.75ex]
%               \end{tabularx}
%               }
%          }
%
%		\newcommand{\figtext}[1]{
%		     %\vspace{-1.9ex}
%		     \captionsetup{justification=justified,font=footnotesize}
%		     \caption*{\hspace{6pt}\hangindent=1.5em #1}
%		     }
%		\newcommand{\fignote}[1]{\figtext{\emph{Note:~}~#1}}
%
\usepackage{collcell}
%\makeatother
% \newcolumntype{G}{>{\collectcell\@gobble}c<{\endcollectcell}@{}}
% \makeatother
% \def\eatcell#1\unskip{}
% \newcolumntype{E}{>{\eatcell}c@{}}
%\usepackage{tabulary}
%\usepackage{multirow}
%\usepackage{dcolumn}
%\usepackage{pdflscape}
%\usepackage{pdfpages}
% \usepackage{epsfig}
% \usepackage{epstopdf}
% \usepackage{eso-pic}
\usepackage{graphicx}
%\usepackage{arydshln}
\usepackage[compatibility=false,font={sc,rm,color=blue},justification=centering,labelformat=empty, textfont=Large, margin=2pt]{caption}
\captionsetup[figure]{belowskip=0pt}

\newcommand{\rot}[2]{\rule{1em}{0pt}%
\makebox[0cm][c]{\rotatebox{#1}{\ #2}}}

\usepackage{siunitx} %For aligning decimals
\sisetup{ detect-mode, 
          group-digits            = false ,
          input-signs             = ,
          input-symbols           = ()[]-+* ,
          input-open-uncertainty  = ,
          input-close-uncertainty = ,
          table-align-text-post   = false, 
          table-number-alignment = center
}
\selectcolormodel{cmyk}
\usepackage{color,soul}
\usepackage{colortbl}
\usepackage{tikz}
\usetikzlibrary{matrix,shapes,arrows,intersections,calc}
\usepackage{verbatim}
\setbeamercovered{invisible}
\setbeamercolor{math text displayed}{fg=blue}
\setbeamercolor{math text inlined}{fg=blue}

%\let\olditem\item
%\renewcommand{\item}{\setlength{\itemsep}{\fill}\olditem}
\AtBeginDocument{\setlength\belowdisplayskip{0pt}}


\usepackage[english]{babel}
\usepackage{booktabs}
\usepackage{tablefootnote}
\usepackage{calc,hhline,ifthen,lscape} 

%\usepackage{enumitem}
%\let\olditem\item
%\renewcommand{\item}{\setlength{\itemsep}{\fill}\olditem}

% new math commands
\newcommand{\E}{\mathbb{E}}

\newcommand{\sym}[1]{\rlap{$#1$}} %For sym in STATA tables

\setbeamertemplate{frametitle}[default][center]

% \makeglossaries
% 
% \usepackage{pgfpages}
% \pgfpagesuselayout{resize to}[a4paper, landscape, border shrink=5mm]
\usepackage[absolute,overlay]{textpos}

\usepackage{epstopdf}


%\setlength{\itemsep}{\fill}



% ===========================================================
% ===========================================================
% ===========================================================
% Improves spacing of itemize and enumerate environment

\makeatletter
\renewcommand{\itemize}[1][]{%
  \beamer@ifempty{#1}{}{\def\beamer@defaultospec{#1}}%
  \ifnum \@itemdepth >2\relax\@toodeep\else
    \advance\@itemdepth\@ne
    \beamer@computepref\@itemdepth% sets \beameritemnestingprefix
    \usebeamerfont{itemize/enumerate \beameritemnestingprefix body}%
    \usebeamercolor[fg]{itemize/enumerate \beameritemnestingprefix body}%
    \usebeamertemplate{itemize/enumerate \beameritemnestingprefix body begin}%
    \list
      {\usebeamertemplate{itemize \beameritemnestingprefix item}}
      {\def\makelabel##1{%
          {%
            \hss\llap{{%
                \usebeamerfont*{itemize \beameritemnestingprefix item}%
                \usebeamercolor[fg]{itemize \beameritemnestingprefix item}##1}}%
          }%
        }%
      }
  \fi%
  \setlength\itemsep{\fill}
    \ifnum \@itemdepth >1
        \vfill
    \fi%  
  \beamer@cramped%
  \raggedright%
  \beamer@firstlineitemizeunskip%
}

\def\enditemize{\ifhmode\unskip\fi\endlist%
  \usebeamertemplate{itemize/enumerate \beameritemnestingprefix body end}
  \ifnum \@itemdepth >1
        \vfil
  \fi%  
  }
\makeatother


\makeatletter
\def\enumerate{%
	\ifnum\@enumdepth>2\relax\@toodeep
	\else%
	\advance\@enumdepth\@ne%
	\edef\@enumctr{enum\romannumeral\the\@enumdepth}%
	\advance\@itemdepth\@ne%
	\fi%
	\beamer@computepref\@enumdepth% sets \beameritemnestingprefix
	\edef\beamer@enumtempl{enumerate \beameritemnestingprefix item}%
	\@ifnextchar[{\beamer@@enum@}{\beamer@enum@}}
\def\beamer@@enum@[{\@ifnextchar<{\beamer@enumdefault[}{\beamer@@@enum@[}}
\def\beamer@enumdefault[#1]{\def\beamer@defaultospec{#1}%
	\@ifnextchar[{\beamer@@@enum@}{\beamer@enum@}}
\def\beamer@@@enum@[#1]{% partly copied from enumerate.sty
	\@enLab{}\let\@enThe\@enQmark
	\@enloop#1\@enum@
	\ifx\@enThe\@enQmark\@warning{The counter will not be printed.%
		^^J\space\@spaces\@spaces\@spaces The label is: \the\@enLab}\fi
	\def\insertenumlabel{\the\@enLab}
	\def\beamer@enumtempl{enumerate mini template}%
	\expandafter\let\csname the\@enumctr\endcsname\@enThe
	\csname c@\@enumctr\endcsname7
	\expandafter\settowidth
	\csname leftmargin\romannumeral\@enumdepth\endcsname
	{\the\@enLab\hspace{\labelsep}}%
	\beamer@enum@}
\def\beamer@enum@{%
	\beamer@computepref\@itemdepth% sets \beameritemnestingprefix
	\usebeamerfont{itemize/enumerate \beameritemnestingprefix body}%
	\usebeamercolor[fg]{itemize/enumerate \beameritemnestingprefix body}%
	\usebeamertemplate{itemize/enumerate \beameritemnestingprefix body begin}%
	\expandafter
	\list
	{\usebeamertemplate{\beamer@enumtempl}}
	{\usecounter\@enumctr%
		\def\makelabel##1{{\hss\llap{{%
						\usebeamerfont*{enumerate \beameritemnestingprefix item}%
						\usebeamercolor[fg]{enumerate \beameritemnestingprefix item}##1}}}}}%
	\setlength\itemsep{\fill}
	\ifnum \@itemdepth >1
	\vfill
	\fi%  
	\beamer@cramped%
	\raggedright%
	\beamer@firstlineitemizeunskip%
}
\def\endenumerate{\ifhmode\unskip\fi\endlist%
	\usebeamertemplate{itemize/enumerate \beameritemnestingprefix body end}
	\ifnum \@itemdepth >1
	\vfil
	\fi%  
}
\makeatother

% ===========================================================
% ===========================================================
% ===========================================================


%\usepackage[colorlinks=true]{hyperref}

\hypersetup{colorlinks = true,linkcolor = blue, bookmarksopen=true, bookmarksopenlevel=1}

%\hypersetup{bookmarksopen=true, bookmarksopenlevel=1}



\begin{document}

\title{The Sequence Space}
\vspace{1cm}
\author[shortname]{
\begin{tabular}{cc}
Juan Herre\~{n}o & Johannes Wieland \\ 
\end{tabular}\\
}



\date{UCSD, Spring \the\year}

\setbeamertemplate{footline}{}
\makebeamertitle
\setbeamertemplate{footline}[frame number]{}

\addtocounter{framenumber}{-1}



%\begin{frame}
%\frametitle[alignment=center]{Reminders}
%\begin{enumerate}
%	\item First project draft due May 1.
%	\item Participation.
%\end{enumerate}
%\end{frame}


%%%%%%%%%%%%%%%%%%%%%%%%%%%%%%%%%%%%%%%%%%%%%%%%%%
\AtBeginSection[]{
\setbeamertemplate{footline}{}
  \frame<beamer>{ 

    \frametitle{Outline}   

    \tableofcontents[currentsection,hideallsubsections] 
  }
\setbeamertemplate{footline}[frame number]{}
\addtocounter{framenumber}{-1}
}

\AtBeginSubsection[]{
\setbeamertemplate{footline}{}
  \frame<beamer>{ 

    \frametitle{Outline}   

    \tableofcontents[currentsection,currentsubsection] 
  }
  \setbeamertemplate{footline}[frame number]{}
  \addtocounter{framenumber}{-1}
}



\setbeamertemplate{footline}{}
\begin{frame}
\frametitle{Outline}   
\tableofcontents[hideallsubsections] 
\end{frame}
\addtocounter{framenumber}{-1}
\setbeamertemplate{footline}[frame number]{}


%%%%%%%%%%%%%%%%%%%%%%%%%%%%%%%%%%%%%%%%%%%%%%%%%%
\section{Introduction}
%%%%%%%%%%%%%%%%%%%%%%%%%%%%%%%%%%%%%%%%%%%%%%%%%%


\begin{frame}
    \frametitle{Solving Models}
    \begin{itemize}
        \item Hard.
        \item Especially when there is heterogeneity.
        \item Infinitely dimensional state space from the distribution of agent's endogenous state variables---how to handle it?
        \item In macro, we are often interested what happens when aggregate shock X hits the economy (monetary, fiscal, etc).
        \item Conditional on the shock, the economy follows a particular \emph{sequence} (path).
%        \item Heterogeneity generally only affects the equilibrium path through its effect on aggregate prices / quantities (real rate, income), which depends on specific moments of the distribution. 
        \item[$\Rightarrow$] Solution only requires knowledge of how heterogeneity evolves along that path (sequence space), not for any other contingencies (state space). 
       	\item Cost: linearization / perfect foresight.
    \end{itemize}
\end{frame}


\begin{frame}
    \frametitle{Demand Supply Example}
    \begin{itemize}
        \item Demand and Supply
        \begin{align*}
            q_i^D &= - \eta^D p + \mu^D q + v^D  \\
            q_j^S &= \eta^S p + \mu^S q + v^S 
        \end{align*}
        \begin{itemize}
            \item Demand/supply elasticities $\eta$, 
            \item ``Agglomoration'' elasticities $\mu$
        \end{itemize}
        % \item Excess Demand:
        % \begin{align*}
        %     q^D - q^S &= - (\eta^D + \eta^S) p + ( v^D - v^S )
        % \end{align*}
        \item Market clearing:
        \begin{align*}
%            p &= \frac{1}{( \frac{\eta^D}{1-\mu^D} + \frac{\eta^S}{1-\mu^S})}\left( \frac{1}{1-\mu^D}v^D -  \frac{1}{1-\mu^S}v^S \right) \\
            q &= \frac{\frac{\eta^S}{1-\mu^S}}{( \frac{\eta^D}{1-\mu^D} + \frac{\eta^S}{1-\mu^S})}\frac{1}{1-\mu^D}v^D  + \frac{\frac{\eta^D}{1-\mu^D} }{( \frac{\eta^D}{1-\mu^D} + \frac{\eta^S}{1-\mu^S})} \frac{1}{1-\mu^S}v^S
        \end{align*}
        \item Sequence space methods are combining demand / supply elasticities and multiplier effects to solve for equilibrium outcomes.
        \item New: these are intertemporal objects.
    \end{itemize}
\end{frame}

%%%%%%%%%%%%%%%%%%%%%%%%%%%%%%%%%%%%%%%%%%%%%%%%%%
\section{NK Example}
%%%%%%%%%%%%%%%%%%%%%%%%%%%%%%%%%%%%%%%%%%%%%%%%%%

\begin{frame}
    \frametitle{NK Model}
    \begin{itemize}
        \item Simple NK model with perfect forsight:
        \begin{align*}
            y_t &= -\frac{1}{\sigma}(i_t - \pi_{t+1}) + \E_t y_{t+1}  \\
            \pi_t &= \kappa (y_t - y_t^{flex}) + \beta\pi_{t+1}
        \end{align*}
        \begin{itemize}
            \item Exogenous $i_t, y_t^{flex}$ (determinacy?) 
            \item Perfect foresight.
        \end{itemize}
\end{itemize}
\end{frame}

\begin{frame}
    \frametitle{Writing in Sequence Space}
    \begin{itemize}
        \item Equations hold at every point in time starting at $t=0$. 
        \item First stack Euler equations for $t=0,1,...,T$:
        \begin{align*}
        	\begin{pmatrix}
        		1 & -1 & 0 & \hdots & 0 \\
        		0 & 1 & -1 & 0 & \vdots \\
        		\vdots & 0 & \ddots & \ddots & 0 \\
        		0 & \hdots & 0 & 1 & -1 \\
        	\end{pmatrix}
        	\begin{pmatrix}
        		y_0 \\
        		y_1 \\
        		\vdots \\
        		y_T \\
        	\end{pmatrix}
        	=&-\begin{pmatrix}
        		1 & 0 & \hdots & 0  \\
        		0 & 1 & 0 & \vdots  \\
        		\vdots & 0 & \ddots & 0  \\
        		0 & \hdots & 0 & 1  \\
        	\end{pmatrix}
        	\begin{pmatrix}
        		i_0 \\
        		i_1 \\
        		\vdots \\
        		i_T \\
        	\end{pmatrix}
        	\\
        	&+ 
        	\begin{pmatrix}
        		0 & 1 & 0 & \hdots & 0 \\
        		0 & 0 & 1 & \hdots & 0 \\
        		\vdots & 0 & \ddots & \ddots & 1 \\
        		0 & \vdots & 0 & 0 & 0 \\
        	\end{pmatrix}
        	\begin{pmatrix}
        		\pi_0 \\
        		\pi_1 \\
        		\vdots \\
        		\pi_T \\
        	\end{pmatrix}
        \end{align*}
%        \begin{itemize}
%        	\item Time periods are stacked in rows.
%        \end{itemize}
    \end{itemize}
\end{frame}

\begin{frame}
    \frametitle{Writing in Sequence Space}
    \begin{itemize}
        \item Then stack the NKPC:
        \begin{align*}
        	\begin{pmatrix}
        		1 & -\beta & 0 & \hdots & 0 \\
        		0 & 1 & -\beta & 0 & \vdots \\
        		\vdots & 0 & \ddots & \ddots & 0 \\
        		0 & \hdots & 0 & 1 & -\beta \\
        	\end{pmatrix}
        	\begin{pmatrix}
        		\pi_0 \\
        		\pi_1 \\
        		\vdots \\
        		\pi_T \\
        	\end{pmatrix}
        	=&\begin{pmatrix}
        		\kappa & 0 & \hdots & 0  \\
        		0 & \kappa & 0 & \vdots  \\
        		\vdots & 0 & \ddots & 0  \\
        		0 & \hdots & 0 & \kappa  \\
        	\end{pmatrix}
        	\begin{pmatrix}
        		y_0 \\
        		y_1 \\
        		\vdots \\
        		y_T \\
        	\end{pmatrix}
        	\\
        	&-\begin{pmatrix}
        		\kappa & 0 & \hdots & 0  \\
        		0 & \kappa & 0 & \vdots  \\
        		\vdots & 0 & \ddots & 0  \\
        		0 & \hdots & 0 & \kappa  \\
        	\end{pmatrix}
        	\begin{pmatrix}
        		y_0^{flex} \\
        		y_1^{flex} \\
        		\vdots \\
        		y_T^{flex} \\
        	\end{pmatrix}
        \end{align*}
        \item In matrix form:
        \begin{align*}
            \bm{\Phi}_Y \bm{Y} &=  -\bm{\Phi}_i \bm{i}  +  \bm{\Phi}_\pi \bm{\pi}\\
            \bm{\Psi}_\pi \bm{\pi} &=  \bm{\Psi}_Y \bm{Y}  -  \bm{\Psi}_Y \bm{Y}^{flex} 
        \end{align*}
    \end{itemize}
\end{frame}


\begin{frame}
    \frametitle{Solution is an Elasticity Formula}
    \begin{itemize}
    	\item Define $\bm{\eta}_\pi^D\equiv -\bm{\Phi}_Y^{-1}\bm{\Phi}_\pi$, $\bm{\eta}_\pi^S\equiv \bm{\Psi}_Y^{-1} \bm{\Psi}_\pi$, $\eta_i^D = \bm{\Phi}_Y^{-1}\bm{\Phi}_i$:
        \begin{align*}
             \bm{Y} &=  - \bm{\eta}_i^D \bm{i}  -  \bm{\eta}_\pi^D \bm{\pi}\\
            \bm{\eta}_\pi^{S} \bm{\pi} &=   \bm{Y}  - \bm{Y}^{flex} 
        \end{align*}
        \item Solution:
        \begin{align*}
             \bm{Y} &= (\bm{\eta}_\pi^{S}) [\bm{\eta}_\pi^{S} +  \bm{\eta}_\pi^{D} ]^{-1}  [- \bm{\eta}_i^D \bm{i}  +  \bm{\eta}_\pi^{D} (\bm{\eta}_\pi^{S})^{-1} \bm{Y}^{flex}] 
        \end{align*}
        \item Interpretation:
        \begin{itemize}
        	\item $ - \bm{\eta}_i^D \bm{i}  +  \bm{\eta}_\pi^{D} (\bm{\eta}_\pi^{S})^{-1} \bm{Y}^{flex}$ is the effect holding on output without the endogenous feedback through inflation.
        	\item $(\bm{\eta}_\pi^{S}) [\bm{\eta}_\pi^{S} +  \bm{\eta}_\pi^{D} ]^{-1}$ captures the feedback loop of inflation and output through demand / supply elasticities:
%        	\begin{align*}
%        		[\bm{I} -  \bm{\Phi}_Y^{-1}\bm{\Phi}_\pi \bm{\Psi}_\pi^{-1} \bm{\Psi}_Y]^{-1} = \bm{\Psi}_Y^{-1} \bm{\Psi}_\pi  [\bm{\Psi}_Y^{-1} \bm{\Psi}_\pi  -  \bm{\Phi}_Y^{-1}\bm{\Phi}_\pi ]^{-1}
%        	\end{align*}
%        	\begin{itemize}
%        		\item $-\bm{\Phi}_Y^{-1}\bm{\Phi}_\pi$ is the demand elasticity: how does a change in price affect demand.
%        		\item $ \bm{\Psi}_Y^{-1} \bm{\Psi}_\pi$ is the supply elasticity: how does a change in price affect supply.
%        	\end{itemize}
        \end{itemize}
    \end{itemize}
\end{frame}

\begin{frame}
    \frametitle{Intertemporal Supply and Demand}
    \begin{itemize}
        \item Solution:
        \begin{align*}
             \bm{Y} &= (\bm{\eta}_\pi^{S}) [\bm{\eta}_\pi^{S} +  \bm{\eta}_\pi^{D} ]^{-1}  [- \bm{\eta}_i^D \bm{i}  +  \bm{\eta}_\pi^{D} (\bm{\eta}_\pi^{S})^{-1} \bm{Y}^{flex}] 
        \end{align*}
        \item If supply elasticities very small:
        \begin{align*}
        	\bm{Y} & \approx \bm{Y}^{flex}
        \end{align*}
        \item If supply elasticities very large:
        \begin{align*}
        	\bm{Y} & \approx - \bm{\eta}_i^D \bm{i}
        \end{align*}
        \item If demand elasticities very large:
        \begin{align*}
        	\bm{Y} & \approx \bm{Y}^{flex} - (\bm{\eta}_\pi^{S})(\bm{\eta}_\pi^{D})^{-1} \bm{\eta}_\pi^{D} \bm{i}
        \end{align*}
        \item If demand elasticities very small:
        \begin{align*}
        	\bm{Y} & \approx \bm{0}
        \end{align*}
        \item[$\Rightarrow$] All we are doing is intertemporal demand and supply.
    \end{itemize}
\end{frame}


\begin{frame}
    \frametitle{Takeaway}
    \begin{itemize}
        \item Solving linear models with sequence space is linear algebra.
        \item Useful to think in terms of demand and supply elasticities in interpreting model output.
        \item Don't need partial equilibrium or heterogeneity to use sequence space.
        \item Do need that the solution to the model is a sequence.
\end{itemize}
\end{frame}

%%%%%%%%%%%%%%%%%%%%%%%%%%%%%%%%%%%%%%%%%%%%%%%%%%
\section{Consumption Problem}
%%%%%%%%%%%%%%%%%%%%%%%%%%%%%%%%%%%%%%%%%%%%%%%%%%


\begin{frame}
    \frametitle{ARS Consumption function}
    \begin{itemize}
        \item Write down consumption problem in ARS (2018)
\end{itemize}
\end{frame}


\begin{frame}
    \frametitle{New Keynesian Model in Sequence Space}
    \begin{itemize}
        \item Consumption function:
        \begin{align*}
            C_t &= \mathbb{C}(\{Y_{t+s}-T_{t+s},r_{t+s}\}_{s=0}^{\infty}) \\
            &\equiv \mathbb{C}(\bm{Y}-\bm{T},\bm{r})
        \end{align*}
        \item NKPC:
        \begin{align*}
            \pi_t &= \mathbb{S}(\bm{Y} - \bm{Y}^*) 
        \end{align*}
        \item Interest rate rule:
        \begin{align*}
            \bm{r} &= \bm{\Phi}_Y (\bm{Y} - \bm{Y}^*) +  \bm{\Phi}_\pi \bm{\pi} + \bm{\epsilon}^r 
        \end{align*}
        \item Excess demand:
        \begin{align*}
%            \begin{pmatrix} ED_t \\ ED_{t+1} \\ \vdots \end{pmatrix}= 
            \bm{ED} &= \pmb{\mathbb{C}}(\bm{Y}-\bm{T},\bm{r}) + \bm{G} - \bm{Y} =\bm{0}
        \end{align*}
%        \item In equilibrium, $\bm{ED}=\bm{0}$.
    \end{itemize}
\end{frame}


\begin{frame}
    \frametitle{Linearize to Solve with Linear Algebra}
    \begin{itemize}
        \item Linearized Model:
        \begin{align*}
            \bm{\hat{C}} &= (\nabla_{\bm{Y}}\pmb{\mathbb{C}}) (\bm{\hat{Y}} - \bm{\hat{T}}) + (\nabla_{\bm{r}}\pmb{\mathbb{C}})\bm{\hat{r}} \\
            \bm{\hat{\pi}} &= (\nabla_{\bm{Y}}\pmb{\mathbb{S}})(\bm{\hat{Y}} - \bm{\hat{Y}}^*)  \\
            \bm{\hat{r}} &= \bm{\Phi}_Y (\bm{\hat{Y}} - \bm{\hat{Y}}^*) +  \bm{\Phi}_\pi \bm{\hat{\pi}} + \bm{\epsilon}^r 
        \end{align*}
        where
        \begin{align*}
            \nabla_{\bm{Y}}\pmb{\mathbb{C}} = 
            \begin{pmatrix}
                \frac{\partial C_{0}}{\partial Y_0} & \frac{\partial C_{0}}{\partial Y_1} & \hdots \\
                \frac{\partial C_{1}}{\partial Y_0} & \frac{\partial C_{1}}{\partial Y_1} & \hdots \\
                \frac{\partial C_{2}}{\partial Y_0} & \frac{\partial C_{2}}{\partial Y_1} & \hdots \\
                \vdots & \vdots & \ddots \\
            \end{pmatrix}
        \end{align*}
        \item Equilibrium:
         \begin{align*}
             \bm{\hat{Y}} &=  [ \bm{\Phi}_Y  +  \bm{\Phi}_\pi (\nabla_{\bm{Y}}\pmb{\mathbb{S}}) ]^{-1} \{[ \bm{\Phi}_Y  +  \bm{\Phi}_\pi (\nabla_{\bm{Y}}\pmb{\mathbb{S}}) ]^{-1} - [I - \nabla_{\bm{Y}}\pmb{\mathbb{C}}]^{-1}(\nabla_{\bm{r}}\pmb{\mathbb{C}})  \}^{-1}  \times\\ 
             &\times[I - \nabla_{\bm{Y}}\pmb{\mathbb{C}}]^{-1} [\bm{\hat{G}} - (\nabla_{\bm{Y}}\pmb{\mathbb{C}})\bm{\hat{T}} + (\nabla_{\bm{r}}\pmb{\mathbb{C}})\bm{\epsilon}^r ]  
         \end{align*}
    \end{itemize}
\end{frame}


\begin{frame}
    \frametitle{Demand and Supply Elasticities Determine GE}
    \begin{itemize}
        \item Define:
        \begin{itemize}
            \item Demand multiplier $\bm{\mu}^D \equiv [I - \nabla_{\bm{Y}}\pmb{\mathbb{C}}]^{-1}$
            \item Demand elasticity $\bm{\eta^D} \equiv -\nabla_{\bm{r}}\pmb{\mathbb{C}}$
            \item Supply elasticity $\bm{\eta^S} \equiv [ \bm{\Phi}_Y  +  \bm{\Phi}_\pi (\pmb{\nabla_{\bm{Y}}\pmb{\mathbb{S}}}) ]^{-1}$
        \end{itemize}
    \end{itemize}
    \begin{align*}
        \bm{\hat{Y}} &=  \underbrace{\bm{\eta^S} [ \bm{\eta^S} + \bm{\mu}^D\bm{\eta^D}  ]^{-1}}_{\text{Demand Incidence}}  \bm{\mu}^D  \underbrace{[\bm{\hat{G}} - (\nabla_{\bm{Y}}\pmb{\mathbb{C}})\bm{\hat{T}} + \bm{\eta^D}\bm{\epsilon}^r ]}_{\text{PE Excess Demand}}  \\
        & + \underbrace{\bm{\mu}^D\bm{\eta^D} [ \bm{\eta^S} + \bm{\mu}^D\bm{\eta^D}  ]^{-1}  }_{\text{Supply Incidence}}   \underbrace{\bm{\hat{Y}}^*}_{\text{PE Excess Supply}}  
    \end{align*}
    Implications:
    \begin{itemize}
        \item GE effects determined by a standard incidence formula.  
        \item[$\Rightarrow$] Matrices of micro elasticities are sufficient statistics for GE effects. 
%        \\
%        Auclert, Rognlie, Straub (2018, 2020a, 2020b), Auclert, Bardoczy, Rognlie, Straub (2020)
%        \item[$\Rightarrow$] Model tells us which PE elasticities matter.
    \end{itemize}
\end{frame}


\begin{frame}
    \frametitle{Balanced Budget Multiplier}
    \begin{itemize}
        \item Balanced budget: $\bm{\hat{G}}$ = $\bm{\hat{T}}$
        \item Also assume no real rate change, so $\bm{\eta^S}$ blows up in some meaningful sense.
        \item Then,
         \begin{align*}
        \bm{\hat{Y}} &=    \bm{\mu}^D  [\bm{\hat{G}} - (\nabla_{\bm{Y}}\pmb{\mathbb{C}})\bm{\hat{T}}] \\
                & = \bm{\mu}^D  [\bm{I} - (\nabla_{\bm{Y}}\pmb{\mathbb{C}})]\bm{\hat{G}}  \\
                &= \bm{\hat{G}}
    \end{align*}
    \item The balanced budget multiplier is 1.
    \end{itemize}
\end{frame}



\end{document}