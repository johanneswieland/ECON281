\documentclass[english,xcolor=svgnames]{beamer}


\usepackage{mathptmx}
\usepackage[OT1]{fontenc}
% \usepackage[latin9]{inputenc}
\usepackage{amsmath}
\usepackage{amssymb}
\usepackage{amsthm}
\usepackage{mathrsfs}
\usepackage{amsfonts}
\usepackage{eurosym}
\usepackage{bm}

\usepackage{booktabs}
\usepackage{tabularx}
\usepackage{subcaption}
\usepackage[makeroom,thicklines]{cancel}

\usepackage{multirow}
\usepackage{rotating}
\usepackage{array}
\usepackage{float}



\makeatletter

 \newcommand\makebeamertitle{\frame{\maketitle}}%
 \AtBeginDocument{
   \let\origtableofcontents=\tableofcontents
 \def\tableofcontents{\@ifnextchar[{\origtableofcontents}{\gobbletableofcontents}}
   \def\gobbletableofcontents#1{\origtableofcontents}
 }
 
 \usetheme{Boadilla}
\setbeamertemplate{footline}[frame number]{}
\usefonttheme{structuresmallcapsserif}
\setbeamercolor{title}{fg=blue}
\setbeamercolor{frametitle}{fg=blue}
\setbeamercolor{caption name}{fg=blue}
\setbeamercovered{transparent}


\beamertemplatenavigationsymbolsempty

\usepackage{booktabs}
\usepackage{tabularx}
\renewcommand{\tabularxcolumn}[1]{>{\centering\arraybackslash}m{#1}}
%\newcolumntype{L}{>{\centering}X}
%\newcolumntype{H}{>{\lrbox0}c<{\endlrbox}@{}}

%\let\estinput=\input
%\newcommand{\estwide}[3]{
%          \vspace{.75ex}{
%               \begin{tabularx}
%               {\textwidth}{@{\hskip\tabcolsep\extracolsep\fill}l*{#2}{#3}}
%               \toprule
%               \estinput{#1}
%               \bottomrule
%               \addlinespace[.75ex]
%               \end{tabularx}
%               }
%          }
%
%		\newcommand{\figtext}[1]{
%		     %\vspace{-1.9ex}
%		     \captionsetup{justification=justified,font=footnotesize}
%		     \caption*{\hspace{6pt}\hangindent=1.5em #1}
%		     }
%		\newcommand{\fignote}[1]{\figtext{\emph{Note:~}~#1}}
%
\usepackage{collcell}
%\makeatother
% \newcolumntype{G}{>{\collectcell\@gobble}c<{\endcollectcell}@{}}
% \makeatother
% \def\eatcell#1\unskip{}
% \newcolumntype{E}{>{\eatcell}c@{}}
%\usepackage{tabulary}
%\usepackage{multirow}
%\usepackage{dcolumn}
%\usepackage{pdflscape}
%\usepackage{pdfpages}
% \usepackage{epsfig}
% \usepackage{epstopdf}
% \usepackage{eso-pic}
\usepackage{graphicx}
%\usepackage{arydshln}
\usepackage[compatibility=false,font={sc,rm,color=blue},justification=centering,labelformat=empty, textfont=Large, margin=2pt]{caption}
\captionsetup[figure]{belowskip=0pt}

\newcommand{\rot}[2]{\rule{1em}{0pt}%
\makebox[0cm][c]{\rotatebox{#1}{\ #2}}}

\usepackage{siunitx} %For aligning decimals
\sisetup{ detect-mode, 
          group-digits            = false ,
          input-signs             = ,
          input-symbols           = ()[]-+* ,
          input-open-uncertainty  = ,
          input-close-uncertainty = ,
          table-align-text-post   = false, 
          table-number-alignment = center
}
\selectcolormodel{cmyk}
\usepackage{color,soul}
\usepackage{colortbl}
\usepackage{tikz}
\usetikzlibrary{matrix,shapes,arrows,intersections,calc}
\usepackage{verbatim}
\setbeamercovered{invisible}
\setbeamercolor{math text displayed}{fg=blue}
\setbeamercolor{math text inlined}{fg=blue}

%\let\olditem\item
%\renewcommand{\item}{\setlength{\itemsep}{\fill}\olditem}
\AtBeginDocument{\setlength\belowdisplayskip{0pt}}


\usepackage[english]{babel}
\usepackage{booktabs}
\usepackage{tablefootnote}
\usepackage{calc,hhline,ifthen,lscape} 

%\usepackage{enumitem}
%\let\olditem\item
%\renewcommand{\item}{\setlength{\itemsep}{\fill}\olditem}

% new math commands
\newcommand{\E}{\mathbb{E}}

\newcommand{\sym}[1]{\rlap{$#1$}} %For sym in STATA tables

\setbeamertemplate{frametitle}[default][center]

% \makeglossaries
% 
% \usepackage{pgfpages}
% \pgfpagesuselayout{resize to}[a4paper, landscape, border shrink=5mm]
\usepackage[absolute,overlay]{textpos}

\usepackage{epstopdf}


%\setlength{\itemsep}{\fill}



% ===========================================================
% ===========================================================
% ===========================================================
% Improves spacing of itemize and enumerate environment

\makeatletter
\renewcommand{\itemize}[1][]{%
  \beamer@ifempty{#1}{}{\def\beamer@defaultospec{#1}}%
  \ifnum \@itemdepth >2\relax\@toodeep\else
    \advance\@itemdepth\@ne
    \beamer@computepref\@itemdepth% sets \beameritemnestingprefix
    \usebeamerfont{itemize/enumerate \beameritemnestingprefix body}%
    \usebeamercolor[fg]{itemize/enumerate \beameritemnestingprefix body}%
    \usebeamertemplate{itemize/enumerate \beameritemnestingprefix body begin}%
    \list
      {\usebeamertemplate{itemize \beameritemnestingprefix item}}
      {\def\makelabel##1{%
          {%
            \hss\llap{{%
                \usebeamerfont*{itemize \beameritemnestingprefix item}%
                \usebeamercolor[fg]{itemize \beameritemnestingprefix item}##1}}%
          }%
        }%
      }
  \fi%
  \setlength\itemsep{\fill}
    \ifnum \@itemdepth >1
        \vfill
    \fi%  
  \beamer@cramped%
  \raggedright%
  \beamer@firstlineitemizeunskip%
}

\def\enditemize{\ifhmode\unskip\fi\endlist%
  \usebeamertemplate{itemize/enumerate \beameritemnestingprefix body end}
  \ifnum \@itemdepth >1
        \vfil
  \fi%  
  }
\makeatother


\makeatletter
\def\enumerate{%
	\ifnum\@enumdepth>2\relax\@toodeep
	\else%
	\advance\@enumdepth\@ne%
	\edef\@enumctr{enum\romannumeral\the\@enumdepth}%
	\advance\@itemdepth\@ne%
	\fi%
	\beamer@computepref\@enumdepth% sets \beameritemnestingprefix
	\edef\beamer@enumtempl{enumerate \beameritemnestingprefix item}%
	\@ifnextchar[{\beamer@@enum@}{\beamer@enum@}}
\def\beamer@@enum@[{\@ifnextchar<{\beamer@enumdefault[}{\beamer@@@enum@[}}
\def\beamer@enumdefault[#1]{\def\beamer@defaultospec{#1}%
	\@ifnextchar[{\beamer@@@enum@}{\beamer@enum@}}
\def\beamer@@@enum@[#1]{% partly copied from enumerate.sty
	\@enLab{}\let\@enThe\@enQmark
	\@enloop#1\@enum@
	\ifx\@enThe\@enQmark\@warning{The counter will not be printed.%
		^^J\space\@spaces\@spaces\@spaces The label is: \the\@enLab}\fi
	\def\insertenumlabel{\the\@enLab}
	\def\beamer@enumtempl{enumerate mini template}%
	\expandafter\let\csname the\@enumctr\endcsname\@enThe
	\csname c@\@enumctr\endcsname7
	\expandafter\settowidth
	\csname leftmargin\romannumeral\@enumdepth\endcsname
	{\the\@enLab\hspace{\labelsep}}%
	\beamer@enum@}
\def\beamer@enum@{%
	\beamer@computepref\@itemdepth% sets \beameritemnestingprefix
	\usebeamerfont{itemize/enumerate \beameritemnestingprefix body}%
	\usebeamercolor[fg]{itemize/enumerate \beameritemnestingprefix body}%
	\usebeamertemplate{itemize/enumerate \beameritemnestingprefix body begin}%
	\expandafter
	\list
	{\usebeamertemplate{\beamer@enumtempl}}
	{\usecounter\@enumctr%
		\def\makelabel##1{{\hss\llap{{%
						\usebeamerfont*{enumerate \beameritemnestingprefix item}%
						\usebeamercolor[fg]{enumerate \beameritemnestingprefix item}##1}}}}}%
	\setlength\itemsep{\fill}
	\ifnum \@itemdepth >1
	\vfill
	\fi%  
	\beamer@cramped%
	\raggedright%
	\beamer@firstlineitemizeunskip%
}
\def\endenumerate{\ifhmode\unskip\fi\endlist%
	\usebeamertemplate{itemize/enumerate \beameritemnestingprefix body end}
	\ifnum \@itemdepth >1
	\vfil
	\fi%  
}
\makeatother

% ===========================================================
% ===========================================================
% ===========================================================


%\usepackage[colorlinks=true]{hyperref}

\hypersetup{colorlinks = true,linkcolor = blue, bookmarksopen=true, bookmarksopenlevel=1}

%\hypersetup{bookmarksopen=true, bookmarksopenlevel=1}



\begin{document}

\title{Regional Aggregation II \& Household Aggregation}
\vspace{1cm}
\author[shortname]{
\begin{tabular}{cc}
Juan Herre\~{n}o & Johannes Wieland \\ 
\end{tabular}\\
}



\date{UCSD, Spring \the\year}

\setbeamertemplate{footline}{}
\makebeamertitle
\setbeamertemplate{footline}[frame number]{}

\addtocounter{framenumber}{-1}

%%%%%%%%%%%%%%%%%%%%%%%%%%%%%%%%%%%%%%%%%%%%%%%%%%
\AtBeginSection[]{
\setbeamertemplate{footline}{}
  \frame<beamer>{ 

    \frametitle{Outline}   

    \tableofcontents[currentsection,hideallsubsections] 
  }
\setbeamertemplate{footline}[frame number]{}
\addtocounter{framenumber}{-1}
}

\AtBeginSubsection[]{
\setbeamertemplate{footline}{}
  \frame<beamer>{ 

    \frametitle{Outline}   

    \tableofcontents[currentsection,currentsubsection] 
  }
  \setbeamertemplate{footline}[frame number]{}
  \addtocounter{framenumber}{-1}
}



\setbeamertemplate{footline}{}
\begin{frame}
\frametitle{Outline}   
\tableofcontents[hideallsubsections] 
\end{frame}
\addtocounter{framenumber}{-1}
\setbeamertemplate{footline}[frame number]{}


%%%%%%%%%%%%%%%%%%%%%%%%%%%%%%%%%%%%%%%%%%%%%%%%%%
\section{Introduction}
%%%%%%%%%%%%%%%%%%%%%%%%%%%%%%%%%%%%%%%%%%%%%%%%%%

\begin{frame}
\frametitle[alignment=center]{Monetary Transmission Mechanism}
\begin{itemize}
	\item Intertemporal substitution (changes in the real interest rate affect C and I).
	\item Credit channel: monetary changes affect spreads, ability of banks to make loans, etc. (Jim\'{e}nez, Ongena, Peydr\'{o}, and Saurina, AER 2012)
	\item Relaxing liquidity constraints for some households by raising income (Cloyne, Ferreira, and Surico, ReStud 2020).
	\item Redistribute income to high MPC consumers (Hausman, Rhode, and Wieland, AER 2019).
	\item Increases real money balances (Chodorow-Reich, Gopinath, Mishra, Narayanan, QJE 2019).
\end{itemize}
\end{frame}


%%%%%%%%%%%%%%%%%%%%%%%%%%%%%%%%%%%%%%%%%%%%%%%%%%
\section{Hausman, Rhode, and Wieland (2019, AER)}
%%%%%%%%%%%%%%%%%%%%%%%%%%%%%%%%%%%%%%%%%%%%%%%%%%

\begin{frame}
\frametitle[alignment=center]{Recovery from the Great Depression}
\centering
\includegraphics[scale=0.5]{figures/HRWFIG1.png}
\end{frame}

\begin{frame}
\frametitle[alignment=center]{Large Devaluation from Leaving Gold Standard}
\centering
\includegraphics[scale=0.5]{figures/HRWFIG2.png}
\end{frame}

\begin{frame}
\frametitle[alignment=center]{Tradable Prices Rose}
\centering
\includegraphics[scale=0.5]{figures/HRWFIG3.png}
\end{frame}

\begin{frame}
\frametitle[alignment=center]{Farm Incomes Rose}
\centering
\includegraphics[scale=0.5]{figures/HRWFIG5.png}
\end{frame}


\begin{frame}
\frametitle[alignment=center]{Specification}
\begin{itemize}
	\item Cross-sectional regression of the form:
	\begin{align*}
		\%\Delta \text{Auto sales}_{i,\text{Spring 1933}} = \beta_0 + \beta_1 \text{Agricultural exposure}_i + \gamma'X_i+\epsilon_i
	\end{align*}
	\item What is the identifying assumption?
	\item Comments? Concerns?
\end{itemize}
\end{frame}

\begin{frame}
\frametitle[alignment=center]{Test for Pre-Trends}
\centering
\includegraphics[scale=0.5]{figures/HRWFIG7.png}
\end{frame}

\begin{frame}
\frametitle[alignment=center]{County-level Analysis}
\centering
\includegraphics[scale=0.4]{figures/HRWTAB3.png}
\end{frame}

\begin{frame}
\frametitle[alignment=center]{Convincing?}

\end{frame}

\begin{frame}
\frametitle[alignment=center]{Aggregation Effects?}
\begin{itemize}
	\item Evidence is about \emph{relative} changes in consumption expenditure.
	\item Three mechanisms by which it can be expansionary overall:
	\begin{enumerate}
		\item Redistribution to higher-MPC households.
		\item Improves bank health.
		\item Raises inflation expectations.
	\end{enumerate}
\end{itemize}
\end{frame}

\begin{frame}
\frametitle[alignment=center]{Testing for Differential MPCs}
\begin{itemize}
	\item Cross-sectional regression of the form:
	\begin{align*}
		\%\Delta &\text{Auto sales}_{i,\text{Spring 1933}} =\\
		&  \beta_0 + \beta_1 \Delta\text{farm product value}_i \times \% \text{farms mortgaged} + \\ 
		&+ \beta_2 \text{farm product value}_i \times \% \text{farms mortgaged} \\
		&+ \beta_3 \Delta\text{farm product value}_i  +\beta_4 \% \text{farms mortgaged} \\
		& + \beta_5 \Delta\text{farm product value}_i  + \gamma'X_i+\epsilon_i
	\end{align*}
	\item What is the identifying assumption?
	\item Comments? Concerns?
\end{itemize}
\end{frame}

\begin{frame}
\frametitle[alignment=center]{Debt-Interaction Positiv}
\centering
\includegraphics[scale=0.4]{figures/HRWTAB5a.png}
\end{frame}

\begin{frame}
\frametitle[alignment=center]{Differential Deposit Growth}
\centering
\includegraphics[scale=0.5]{figures/HRWFIG12.png}
\end{frame}

\begin{frame}
\frametitle[alignment=center]{Inflation Expectations?}
\centering
\includegraphics[scale=0.5]{figures/HRWFIG14b.png}
\end{frame}

\begin{frame}
\frametitle{Aggregation}

\begin{itemize}
\item Simple framework to examine how cross-sectional estimates map to the aggregate economy.
\item Model has heterogeneity on the following three dimensions:
\begin{itemize}
	\item Income from farming, labor, or pricing power.
	\item Permanent income vs hand-to-mouth.
	\item Farm vs urban area.
\end{itemize}
\item Simplifications:
\begin{itemize}
	\item Model essentially static.
	\item Exogenous relative price movements.
\end{itemize}
\item Who looked at the appendix?
\end{itemize}

\end{frame}

\begin{frame}
\frametitle{Key result}
\vspace{-0.8cm}
\begin{align*}
	\% \Delta \text{Cars} &= \underbrace{\beta \times
                         \phi^{f}}_{\substack{\text{``naive''} \\
  \text{extrapolation}}} \times  \underbrace{\frac{\text{Farm
  area income per capita}}{\text{National income per
  capita}}}_{\text{Relative income p.c.}} \\
  & \times  \underbrace{\left( 1-\xi\frac{\theta^{w}}{\theta^{f}} \right)}_{\substack{\text{Redistribution from } \\ \text{high-MPC consumers}}} \times \underbrace{\mu_{t}}_{\substack{\text{Aggregate}\\ \text{spending}\\ \text{multiplier}}} \\
  &\qquad  + \underbrace{-\sigma  d\ln (1+r_t)}_{\text{Intertemporal Substitution}}
\end{align*}
\begin{itemize}
	\item Comments? Concerns?
\end{itemize}

\end{frame}

\begin{frame}
\frametitle[alignment=center]{Aggregate Effect of Farm Channel}
\centering
\includegraphics[scale=0.4]{figures/HRWTAB7.png}
\begin{itemize}
	\item Thoughts? Comments?
\end{itemize}
\end{frame}

%%%%%%%%%%%%%%%%%%%%%%%%%%%%%%%%%%%%%%%%%%%%%%%%%%
\section{Cloyne, Ferreira, and Surico (2020, ReStud)}
%%%%%%%%%%%%%%%%%%%%%%%%%%%%%%%%%%%%%%%%%%%%%%%%%%

\begin{frame}
\frametitle[alignment=center]{Data}
\begin{itemize}
	\item Monetary shocks for U.S., U.K.
	\item Consumer expenditure data.
	\begin{itemize}
		\item Detailed data on consumption.
		\item More rudimentary data on income and especially wealth.
		\item Contains information on housing tenure and housing debt status.
	\end{itemize}
\end{itemize}
\end{frame}


\begin{frame}
\frametitle[alignment=center]{Aggregate Monetary Shock}
\centering
\includegraphics[scale=0.3]{figures/CFSFIG2.png}
\begin{itemize}
	\item Thoughts? Comments?
\end{itemize}
\end{frame}

\begin{frame}
\frametitle[alignment=center]{Specification}
\begin{align*}
	X_{i,t} = \alpha_0^i + \alpha_1^i trend + B^i(L)X_{i,t-1} + C^i(L)S_{t-1} + \sum_{q=2}^{4}D_q^i Z_q + u_{i,t}
\end{align*}
\begin{itemize}
	\item $i \in$ [Mortgagor, Outright-Owner, Renter]
	\item Identification assumption?
	\item Comments? Concerns?
\end{itemize}
\end{frame}

\begin{frame}
\frametitle[alignment=center]{Nondurable Expenditure}
\centering
\includegraphics[scale=0.3]{figures/CFSFIG3.png}
\end{frame}

\begin{frame}
\frametitle[alignment=center]{Durable Expenditure}
\centering
\includegraphics[scale=0.3]{figures/CFSFIG4.png}
\end{frame}

\begin{frame}
\frametitle[alignment=center]{Comparison to Income}
\centering
\includegraphics[scale=0.3]{figures/CFSFIG9.png}
\end{frame}

\begin{frame}
\frametitle[alignment=center]{Comparison to Income}
\centering
\includegraphics[scale=0.3]{figures/CFSTAB1.png}
\end{frame}

\begin{frame}
\frametitle[alignment=center]{Demographic Sub-groups}
\centering
\includegraphics[scale=0.3]{figures/CFSFIG6.png}
\end{frame}

\begin{frame}
\frametitle[alignment=center]{Convincing?}

\end{frame}

\begin{frame}
\frametitle[alignment=center]{Questions}
\begin{itemize}
	\item What is their preferred interpretation?
	\item What is causing the rise in income?
\item What are the aggregate implications?
	\item How is it different from Hausman, Rhode, Wieland?
\end{itemize}
\end{frame}



%%%%%%%%%%%%%%%%%%%%%%%%%%%%%%%%%%%%%%%%%%%%%%%%%%
\section{Parker, Souleles, Johnson, and McClelland (2013, AER)}
%%%%%%%%%%%%%%%%%%%%%%%%%%%%%%%%%%%%%%%%%%%%%%%%%%

\begin{frame}
\frametitle[alignment=center]{What is the MPC?}
\begin{itemize}
	\item MPC = marginal propensity to consume.
	\item Very important parameter in old Keynesian models.
	\item In standard New Keynesian models $\approx 0$.
	\begin{itemize}
		\item Euler equation $\Rightarrow$ Permanent income consumer.
	\end{itemize}
	\item TANK and HANK models.
\end{itemize}
\end{frame}

\begin{frame}
\frametitle[alignment=center]{Identification Problem}
\begin{align*}
	c_{it}  = \alpha + \beta y_{it} + \epsilon_{it}
\end{align*}
\begin{itemize}
	\item What could go wrong?
\end{itemize}
\end{frame}

\begin{frame}
\frametitle[alignment=center]{Jonathan Parker Oeuvre}
\begin{itemize}
	\item Johnson, Parker, Souleles, AER 2003: 20-40\% of 2001 Rebate spent on nondurable goods within 3 months.
	\item Parker, Souleles, Johnson, McClelland, AER 2008: 50-90\% of 2008 Rebate spent on nondurable and durable goods within 3 months.
	\item Broda, Parker, JME 2014: 2008 rebate caused 10\% increase in spending in first week.
	\item Parker, Schild, Erhard, Johnson, WP 2022: 10\% of 2020 stimulus was spent within 3 months.
\end{itemize}
\end{frame}

\begin{frame}
\frametitle[alignment=center]{The 2008 Experiment}
\centering
\includegraphics[scale=0.6]{figures/PSMJTAB1.png}
\end{frame}

\begin{frame}
\frametitle[alignment=center]{Specification}
\begin{align*}
	C_{i,t+1}-C_{i,t} = \sum_s \beta_{0s} \times month_{s,i} + \beta_1'X_{i,t} + \beta_2 ESP_{i,t+1} + u_{i,t+1}
\end{align*}
\begin{itemize}
	\item Comments? Concerns?
\end{itemize}
\end{frame}

\begin{frame}
\frametitle[alignment=center]{Effects on Expenditure}
\centering
\includegraphics[scale=0.6]{figures/PSMJTAB2.png}
\end{frame}

\begin{frame}
\frametitle[alignment=center]{Sub-Samples}
\centering
\includegraphics[scale=0.6]{figures/PSMJTAB3.png}
\end{frame}

\begin{frame}
\frametitle[alignment=center]{Persistence}
\centering
\includegraphics[scale=0.6]{figures/PSMJTAB5.png}
\end{frame}


\begin{frame}
\frametitle[alignment=center]{Heterogeneous Treatment Effects}
\centering
\includegraphics[scale=0.6]{figures/PSMJTAB6.png}
\end{frame}


\begin{frame}
\frametitle[alignment=center]{Convincing?}

\end{frame}

\begin{frame}
\frametitle[alignment=center]{More MPCs}
\begin{itemize}
	\item Shapiro and Slemrod (AER 2003, AER, 2009): self-reported MPC of 25-30\% out of rebates in 2001 / 2008.
	\item Japielli and Pistaferri (AEJ-Macro, 2014): self-reported MPC of 48\% out of hypothetical transitory income shock.
	\item Faegereng, Holmn, and Natvick (AEJ-Macro, 2021): 50\% MPC within one year of large lottery winnings in Norway. Consumption is resiaul from budget constraint: $C=Y - \Delta A$.
\end{itemize}
\end{frame}


%%%%%%%%%%%%%%%%%%%%%%%%%%%%%%%%%%%%%%%%%%%%%%%%%%
\section{de Chaisemartin and D'Haultf{\oe}uille (2020, AER)}
%%%%%%%%%%%%%%%%%%%%%%%%%%%%%%%%%%%%%%%%%%%%%%%%%%

\begin{frame}
\frametitle[alignment=center]{de Chaisemartin and D'Haultf{\oe}uille, AER 2020}
\begin{itemize}
	\item Panel, binned into cells $g,t$ (g=group).
	\item $Y_{i,g,t}$ outcome of unit $i$ in cell $g,t$.
	\item $D_{g,t}$ treatment indicator.
	\item Expectation of OLS 2-way FE estimator:
	\begin{align*}
		\beta_{fe} = E \left(\sum_{(g,t):D_{g,t}=1} W_{g,t}\Delta_{g,t}\right)
	\end{align*}
	\begin{itemize}
		\item $W_{g,t}$ are weights, $\sum_{(g,t):D_{g,t}=1} W_{g,t}=1$.
		\item $\Delta_{g,t}$ is the group-specific ATE.
	\end{itemize}
\end{itemize}
\end{frame}

\begin{frame}
\frametitle[alignment=center]{What is the Problem?}
\begin{itemize}
	\item With homogeneous treatment effects, no problem:
	\begin{align*}
		 \Delta_{g,t}=\Delta \;\Rightarrow\; \beta_{fe} = \Delta
	\end{align*}
	\item With heterogenous treatment effects $\beta_{fe}$ may be poor guide to average ATE since weights $W_{g,t}$ may be negative.
\end{itemize}
\end{frame}

%%%%%%%%%%%%%%%%%%%%%%%%%%%%%%%%%%%%%%%%%%%%%%%%%%
\section{Borusyak, Jaravel, and Spiess (2022, WP)}
%%%%%%%%%%%%%%%%%%%%%%%%%%%%%%%%%%%%%%%%%%%%%%%%%%

\begin{frame}
\frametitle[alignment=center]{Borusyak, Jaravel, and Spiess, WP 2022}
\begin{center}
	\includegraphics[scale=0.5]{figures/BJSTAB1.png}
\end{center}
\begin{itemize}
	\item 2-way FE OLS population coefficient is:
	\begin{align*}
		\beta_{fe} = \tau_{A2} + \frac{1}{2}\tau_{B3} - \frac{1}{2}\tau_{A3}
	\end{align*}
	\item Not an ATE!
	\item What is OLS doing here?
%	\item Do we get an ATE if we add a lag?
\end{itemize}
\end{frame}


\begin{frame}
\frametitle[alignment=center]{Test for Pre-Trends}
\begin{center}
	\includegraphics[scale=0.5]{figures/BJSTAB1.png}
\end{center}
\begin{itemize}
	\item Pre-trend coefficient for lag 2:
	\begin{align*}
		\beta_{fe,-2} &= \tau_{A3} - \tau_{B3}
	\end{align*}
	\item What is OLS doing here?
	\item Identified?
\end{itemize}
\end{frame}

\begin{frame}
\frametitle[alignment=center]{Notation}
\begin{itemize}
	\item Binary treatment $D_{it}$, outcome $Y_{it}$
	\item Event date $E_{it}$ where $D_{it}$ switches from 0 to 1.
	\item Observations $\Omega_1=\{it\in \Omega:\; D_{it}=1\}$ and not-yet-treated $\Omega_0$ (includes never treated). 
	\begin{itemize}
		\item Treated: $\Omega_1=\{it\in \Omega:\; D_{it}=1\}$, $|\Omega_1|=N_1$
		\item Not-yet-treated: $\Omega_0=\{it\in \Omega:\; D_{it}=0\}$, $|\Omega_0|=N_0$
	\end{itemize}
	\item $Y_{it}(0)$ potential outcome if never treated.
	\item Causal effect $\tau_{it}=E[Y_{it}-Y_{it}(0)]$.
\end{itemize}
\end{frame}

\begin{frame}
\frametitle[alignment=center]{Start from First Principles}
\begin{itemize}
	\item Estimation target:
	\begin{align*}
		\tau_w = \sum_{it\in\Omega_1} w_{it}\tau_{it} = w'\tau
	\end{align*}
	\item Assumption 1: Parallel trends 
	\begin{align*}
		E[Y_{it}(0)] = \alpha_i + \beta_t \qquad\forall it\in \Omega
	\end{align*}
	\item Assumption 2: No anticipation
	\begin{align*}
		Y_{it}=Y_{it}(0) \qquad\forall it\in \Omega_0
	\end{align*}
	\item Assumption 3': Restricted causal effects
	\begin{align*}
		\tau =\Gamma \theta 
	\end{align*}
	\begin{itemize}
		\item $\theta$ is unknown $N_1-M \times 1$, $\Gamma$ is known $N_1\times (N_1-M$)
		\item $M$ restrictions on treatment effect. $M=N_1-1$ = homogenous effects.
	\end{itemize}
\end{itemize}
\end{frame}

\begin{frame}
\frametitle[alignment=center]{BSJ Theorem 1 [Simplified]}
\begin{itemize}
	\item Suppose Assumptions 1, 2, 3', and 4 [homoscedastic errors] hold. Then among linear unbiased estimators of $\tau_w$, the (unique) efficient estimator $\hat{\tau}_w^*$ can be obtained with the following steps:
	\begin{enumerate}
		\item Estimate $\theta$ by $\hat{\theta}$ from the linear regression
		\begin{align*}
			Y_{it} = \alpha_i + \beta_t + D_{it}\Gamma_{it}'\theta + \epsilon_{it}.
		\end{align*}
		\item Estimate the vector of treatment effects $\tau$ by $\hat{\tau}=\Gamma \hat{\theta}$.
		\item Estimate the target $\tau_t$ by $\hat{\tau}_w^* = w'\hat{\tau}$
	\end{enumerate}
\end{itemize}
\end{frame}

\begin{frame}
\frametitle[alignment=center]{BSJ Theorem 2 [Simplified]}
\begin{itemize}
	\item With unrestricted treatment effects $(M=0$), the unique efficient linear unbiased estimator $\hat{\tau}_w^*$ of $\tau_w$ from Theorem 1 can be obtained via an imputation procedure:
	\begin{enumerate}
		\item Within the untreated observations only ($it\in \Omega_0$), estimate by OLS:
		\begin{align*}
			Y_{it} = \alpha_i + \beta_t + \epsilon_{it}.
		\end{align*}
		\item For each treated observations ($it\in \Omega_1$) with $w_{it}\neq 0$, set  $\hat{Y}_{it}(0) = \hat{\alpha}_i + \hat{\beta}_t$ and $\hat{\tau}_{it} = \hat{Y}_{it} - \hat{Y}_{it}(0)$.
		\item Estimate the target $\tau_w$ by a weighted sum $\hat{\tau}_w^*=w'\hat{\tau}$
	\end{enumerate}
\end{itemize}
\end{frame}


\begin{frame}
\frametitle[alignment=center]{Inference}
\begin{itemize}
	\item Inference problem for treated units:
	\begin{align*}
			Y_{it} = \alpha_i + \beta_t + \tau_{it} + \epsilon_{it}.
		\end{align*}
	\item How to distinguish between unrestricted $\tau_{it}$ and $\epsilon_{it}$?
	\item ``Conservative'' standard errors: impose some homogeneity, so attribute some variance to $\epsilon_{it}$ that belongs to $\tau_{it}$.
	\item Yields asymptotically weakly conservative standard errors.
\end{itemize}
\end{frame}

\begin{frame}
\frametitle[alignment=center]{Pre-trends}
\begin{itemize}
	\item To test for pre-trends augment model for untreated observations with additional pre-determined variables and test that the coefficients are zero.
	\item Does not distort inference conditional on test passing.
	\item What happens if we then include these variables in the regression model? Do we satisfy parallel trends?
\end{itemize}
\end{frame}


\begin{frame}
\frametitle[alignment=center]{Application to Broda and Parker, JME 2014}
\centering
\includegraphics[scale=0.4]{figures/BJSTAB3a.png}
\end{frame}

\begin{frame}
\frametitle[alignment=center]{Dynamic Treatment Effects}
\centering
\includegraphics[scale=0.5]{figures/BJSFIG2b.png}
\end{frame}

\begin{frame}
\frametitle[alignment=center]{Weights}
\centering
\includegraphics[scale=0.5]{figures/BJSFIG3.png}
\end{frame}

%%%%%%%%%%%%%%%%%%%%%%%%%%%%%%%%%%%%%%%%%%%%%%%%%%
\section{Orchard, Ramey, and Wieland (2022, WP)}
%%%%%%%%%%%%%%%%%%%%%%%%%%%%%%%%%%%%%%%%%%%%%%%%%%


\end{document}