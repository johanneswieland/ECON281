\documentclass[english,xcolor=svgnames]{beamer}


\usepackage{mathptmx}
\usepackage[OT1]{fontenc}
% \usepackage[latin9]{inputenc}
\usepackage{amsmath}
\usepackage{amssymb}
\usepackage{amsthm}
\usepackage{mathrsfs}
\usepackage{amsfonts}
\usepackage{eurosym}
\usepackage{bm}

\usepackage{booktabs}
\usepackage{tabularx}
\usepackage{subcaption}
\usepackage[makeroom,thicklines]{cancel}

\usepackage{multirow}
\usepackage{rotating}
\usepackage{array}
\usepackage{float}



\makeatletter

 \newcommand\makebeamertitle{\frame{\maketitle}}%
 \AtBeginDocument{
   \let\origtableofcontents=\tableofcontents
 \def\tableofcontents{\@ifnextchar[{\origtableofcontents}{\gobbletableofcontents}}
   \def\gobbletableofcontents#1{\origtableofcontents}
 }
 
 \usetheme{Boadilla}
\setbeamertemplate{footline}[frame number]{}
\usefonttheme{structuresmallcapsserif}
\setbeamercolor{title}{fg=blue}
\setbeamercolor{frametitle}{fg=blue}
\setbeamercolor{caption name}{fg=blue}
\setbeamercovered{transparent}


\beamertemplatenavigationsymbolsempty

\usepackage{booktabs}
\usepackage{tabularx}
\renewcommand{\tabularxcolumn}[1]{>{\centering\arraybackslash}m{#1}}
%\newcolumntype{L}{>{\centering}X}
%\newcolumntype{H}{>{\lrbox0}c<{\endlrbox}@{}}

%\let\estinput=\input
%\newcommand{\estwide}[3]{
%          \vspace{.75ex}{
%               \begin{tabularx}
%               {\textwidth}{@{\hskip\tabcolsep\extracolsep\fill}l*{#2}{#3}}
%               \toprule
%               \estinput{#1}
%               \bottomrule
%               \addlinespace[.75ex]
%               \end{tabularx}
%               }
%          }
%
%		\newcommand{\figtext}[1]{
%		     %\vspace{-1.9ex}
%		     \captionsetup{justification=justified,font=footnotesize}
%		     \caption*{\hspace{6pt}\hangindent=1.5em #1}
%		     }
%		\newcommand{\fignote}[1]{\figtext{\emph{Note:~}~#1}}
%
\usepackage{collcell}
%\makeatother
% \newcolumntype{G}{>{\collectcell\@gobble}c<{\endcollectcell}@{}}
% \makeatother
% \def\eatcell#1\unskip{}
% \newcolumntype{E}{>{\eatcell}c@{}}
%\usepackage{tabulary}
%\usepackage{multirow}
%\usepackage{dcolumn}
%\usepackage{pdflscape}
%\usepackage{pdfpages}
% \usepackage{epsfig}
% \usepackage{epstopdf}
% \usepackage{eso-pic}
\usepackage{graphicx}
%\usepackage{arydshln}
\usepackage[compatibility=false,font={sc,rm,color=blue},justification=centering,labelformat=empty, textfont=Large, margin=2pt]{caption}
\captionsetup[figure]{belowskip=0pt}

\newcommand{\rot}[2]{\rule{1em}{0pt}%
\makebox[0cm][c]{\rotatebox{#1}{\ #2}}}

\usepackage{siunitx} %For aligning decimals
\sisetup{ detect-mode, 
          group-digits            = false ,
          input-signs             = ,
          input-symbols           = ()[]-+* ,
          input-open-uncertainty  = ,
          input-close-uncertainty = ,
          table-align-text-post   = false, 
          table-number-alignment = center
}
\selectcolormodel{cmyk}
\usepackage{color,soul}
\usepackage{colortbl}
\usepackage{tikz}
\usetikzlibrary{matrix,shapes,arrows,intersections,calc}
\usepackage{verbatim}
\setbeamercovered{invisible}
\setbeamercolor{math text displayed}{fg=blue}
\setbeamercolor{math text inlined}{fg=blue}

%\let\olditem\item
%\renewcommand{\item}{\setlength{\itemsep}{\fill}\olditem}
\AtBeginDocument{\setlength\belowdisplayskip{0pt}}


\usepackage[english]{babel}
\usepackage{booktabs}
\usepackage{tablefootnote}
\usepackage{calc,hhline,ifthen,lscape} 

%\usepackage{enumitem}
%\let\olditem\item
%\renewcommand{\item}{\setlength{\itemsep}{\fill}\olditem}

% new math commands
\newcommand{\E}{\mathbb{E}}

\newcommand{\sym}[1]{\rlap{$#1$}} %For sym in STATA tables

\setbeamertemplate{frametitle}[default][center]

% \makeglossaries
% 
% \usepackage{pgfpages}
% \pgfpagesuselayout{resize to}[a4paper, landscape, border shrink=5mm]
\usepackage[absolute,overlay]{textpos}

\usepackage{epstopdf}


%\setlength{\itemsep}{\fill}



% ===========================================================
% ===========================================================
% ===========================================================
% Improves spacing of itemize and enumerate environment

\makeatletter
\renewcommand{\itemize}[1][]{%
  \beamer@ifempty{#1}{}{\def\beamer@defaultospec{#1}}%
  \ifnum \@itemdepth >2\relax\@toodeep\else
    \advance\@itemdepth\@ne
    \beamer@computepref\@itemdepth% sets \beameritemnestingprefix
    \usebeamerfont{itemize/enumerate \beameritemnestingprefix body}%
    \usebeamercolor[fg]{itemize/enumerate \beameritemnestingprefix body}%
    \usebeamertemplate{itemize/enumerate \beameritemnestingprefix body begin}%
    \list
      {\usebeamertemplate{itemize \beameritemnestingprefix item}}
      {\def\makelabel##1{%
          {%
            \hss\llap{{%
                \usebeamerfont*{itemize \beameritemnestingprefix item}%
                \usebeamercolor[fg]{itemize \beameritemnestingprefix item}##1}}%
          }%
        }%
      }
  \fi%
  \setlength\itemsep{\fill}
    \ifnum \@itemdepth >1
        \vfill
    \fi%  
  \beamer@cramped%
  \raggedright%
  \beamer@firstlineitemizeunskip%
}

\def\enditemize{\ifhmode\unskip\fi\endlist%
  \usebeamertemplate{itemize/enumerate \beameritemnestingprefix body end}
  \ifnum \@itemdepth >1
        \vfil
  \fi%  
  }
\makeatother


\makeatletter
\def\enumerate{%
	\ifnum\@enumdepth>2\relax\@toodeep
	\else%
	\advance\@enumdepth\@ne%
	\edef\@enumctr{enum\romannumeral\the\@enumdepth}%
	\advance\@itemdepth\@ne%
	\fi%
	\beamer@computepref\@enumdepth% sets \beameritemnestingprefix
	\edef\beamer@enumtempl{enumerate \beameritemnestingprefix item}%
	\@ifnextchar[{\beamer@@enum@}{\beamer@enum@}}
\def\beamer@@enum@[{\@ifnextchar<{\beamer@enumdefault[}{\beamer@@@enum@[}}
\def\beamer@enumdefault[#1]{\def\beamer@defaultospec{#1}%
	\@ifnextchar[{\beamer@@@enum@}{\beamer@enum@}}
\def\beamer@@@enum@[#1]{% partly copied from enumerate.sty
	\@enLab{}\let\@enThe\@enQmark
	\@enloop#1\@enum@
	\ifx\@enThe\@enQmark\@warning{The counter will not be printed.%
		^^J\space\@spaces\@spaces\@spaces The label is: \the\@enLab}\fi
	\def\insertenumlabel{\the\@enLab}
	\def\beamer@enumtempl{enumerate mini template}%
	\expandafter\let\csname the\@enumctr\endcsname\@enThe
	\csname c@\@enumctr\endcsname7
	\expandafter\settowidth
	\csname leftmargin\romannumeral\@enumdepth\endcsname
	{\the\@enLab\hspace{\labelsep}}%
	\beamer@enum@}
\def\beamer@enum@{%
	\beamer@computepref\@itemdepth% sets \beameritemnestingprefix
	\usebeamerfont{itemize/enumerate \beameritemnestingprefix body}%
	\usebeamercolor[fg]{itemize/enumerate \beameritemnestingprefix body}%
	\usebeamertemplate{itemize/enumerate \beameritemnestingprefix body begin}%
	\expandafter
	\list
	{\usebeamertemplate{\beamer@enumtempl}}
	{\usecounter\@enumctr%
		\def\makelabel##1{{\hss\llap{{%
						\usebeamerfont*{enumerate \beameritemnestingprefix item}%
						\usebeamercolor[fg]{enumerate \beameritemnestingprefix item}##1}}}}}%
	\setlength\itemsep{\fill}
	\ifnum \@itemdepth >1
	\vfill
	\fi%  
	\beamer@cramped%
	\raggedright%
	\beamer@firstlineitemizeunskip%
}
\def\endenumerate{\ifhmode\unskip\fi\endlist%
	\usebeamertemplate{itemize/enumerate \beameritemnestingprefix body end}
	\ifnum \@itemdepth >1
	\vfil
	\fi%  
}
\makeatother

% ===========================================================
% ===========================================================
% ===========================================================


%\usepackage[colorlinks=true]{hyperref}

\hypersetup{colorlinks = true,linkcolor = blue, bookmarksopen=true, bookmarksopenlevel=1}

%\hypersetup{bookmarksopen=true, bookmarksopenlevel=1}



\begin{document}

\title{Topics in Macroeconomics }
\vspace{1cm}
\author[shortname]{
\begin{tabular}{cc}
Juan Herre\~{n}o & Johannes Wieland \\ 
\end{tabular}\\
}



\date{UCSD, Spring \the\year}

\setbeamertemplate{footline}{}
\makebeamertitle
\setbeamertemplate{footline}[frame number]{}

\addtocounter{framenumber}{-1}

%%%%%%%%%%%%%%%%%%%%%%%%%%%%%%%%%%%%%%%%%%%%%%%%%%
\AtBeginSection[]{
\setbeamertemplate{footline}{}
  \frame<beamer>{ 

    \frametitle{Outline}   

    \tableofcontents[currentsection,hideallsubsections] 
  }
\setbeamertemplate{footline}[frame number]{}
\addtocounter{framenumber}{-1}
}

\AtBeginSubsection[]{
\setbeamertemplate{footline}{}
  \frame<beamer>{ 

    \frametitle{Outline}   

    \tableofcontents[currentsection,currentsubsection] 
  }
  \setbeamertemplate{footline}[frame number]{}
  \addtocounter{framenumber}{-1}
}



\setbeamertemplate{footline}{}
\begin{frame}
\frametitle{Outline}   
\tableofcontents[hideallsubsections] 
\end{frame}
\addtocounter{framenumber}{-1}
\setbeamertemplate{footline}[frame number]{}


%%%%%%%%%%%%%%%%%%%%%%%%%%%%%%%%%%%%%%%%%%%%%%%%%%
\section{Introduction}
%%%%%%%%%%%%%%%%%%%%%%%%%%%%%%%%%%%%%%%%%%%%%%%%%%


\begin{frame}
\frametitle[alignment=center]{Cross-Sectional Regressions}
\begin{align*}
	Y_i = \alpha_i + \beta X_i + \epsilon_i
\end{align*}
\begin{itemize}
	\item Interested in $\beta$.
	\item Can identify $\beta$ if $E( \epsilon_i|X_i)=0$ or suitable instrument with $E( \epsilon_i|Z_i)=0$.
	\item What's the DGP? Two views:
	\begin{enumerate}
		\item $X_i$ / $Z_i$ is quasi-random heterogenous exposure to the same endogenous shock.
		\item $X_i$ / $Z_i$ is heterogenous, endogenous exposure to heterogenous, quasi-random shocks.
	\end{enumerate}
\end{itemize}
\end{frame}


%%%%%%%%%%%%%%%%%%%%%%%%%%%%%%%%%%%%%%%%%%%%%%%%%%
\section{Goldsmith-Pinkham, Sorkin, and Swift, AER 2020}
%%%%%%%%%%%%%%%%%%%%%%%%%%%%%%%%%%%%%%%%%%%%%%%%%%

\begin{frame}
\frametitle[alignment=center]{Bartik: Canonical Example}
\begin{itemize}
	\item Structural equation:
	\begin{align*}
		y_l = \tau +\beta x_l + \epsilon_l
	\end{align*}
	\begin{itemize}
		\item $y_l$: wage growth in area $l$.
		\item $x_l$: employment growth in area $l$.
	\end{itemize}
	\item Identities:
	\begin{align*}
		x_l = \sum_k z_{l,k} g_{l,k}, \qquad\qquad g_{l,k} = g_k + \tilde{g}_{l,k}
	\end{align*}
	\begin{itemize}
		\item $z_{l,k}$: employment share in area $l$ in industry $k$.
		\item $g_{l,k}$: employment growth in area $l$ in industry $k$.
		\item $g_{k}$: national employment growth in industry $k$.
		\item $\tilde{g}_{l,k}$:  idiosyncratic component of employment growth rate.
	\end{itemize}
	\item Bartik (1991) instrument to estimate inverse labor supply elasticity:
	\begin{align*}
		B_l = \sum_k z_{l,k}g_k
	\end{align*}
	\item What is exogenous? Shares? Shocks? Product?
\end{itemize}
\end{frame}

\begin{frame}
\frametitle[alignment=center]{General Setup}
\begin{itemize}
	\item x
\end{itemize}
\end{frame}


%%%%%%%%%%%%%%%%%%%%%%%%%%%%%%%%%%%%%%%%%%%%%%%%%%
\section{Borusyak, Hull, and Jaravel, RESTUD 2022}
%%%%%%%%%%%%%%%%%%%%%%%%%%%%%%%%%%%%%%%%%%%%%%%%%%

%%%%%%%%%%%%%%%%%%%%%%%%%%%%%%%%%%%%%%%%%%%%%%%%%%
\section{Nakamura and Steinsson, AER 2014}
%%%%%%%%%%%%%%%%%%%%%%%%%%%%%%%%%%%%%%%%%%%%%%%%%%

%%%%%%%%%%%%%%%%%%%%%%%%%%%%%%%%%%%%%%%%%%%%%%%%%%
\section{Mian, Rao, and Sufi, QJE 2013}
%%%%%%%%%%%%%%%%%%%%%%%%%%%%%%%%%%%%%%%%%%%%%%%%%%


\end{document}